\chapter{Introduction}

The introduction serves two purposes: \hfill (Not necessarily in this order.)
\begin{compactenum}
  \item To give a high-level \emph{introduction} into the research area and\label{enum:intro:introduction}
  \item to \emph{motivate} your research done within it.\label{enum:intro:motivation}
\end{compactenum}

Point \ref{enum:intro:introduction} is important because you should not assume any technical background in the specific subject matter of your work. Provide a high-level introduction, but avoid technical definitions unless they are used as an easy-to-understand example that doesn't require prior knowledge. The rule of thumb is that you can assume a very basic understanding of the respective (more general) research area, like computer science, engineering, mathematics, and so on -- based on the audience for which the work is conducted. What such an accurate level of abstraction/presentation is might depend on the specific topic of your work. Also consult your supervisor(s) for their opinion(s) and preferences. 

Point \ref{enum:intro:motivation} should make clear why the investigated research question is worth being investigated -- why is it relevant and important? This is the part where you should make the readers look forward reading your work! Make them interested and passionate!

Be specific about the precise contributions that your work actually does and list them in the text (preferably in a separate paragraph; depending on the number of contributions and how well they can be separated you could also provide them in a bullet point list).

An introduction is usually between 1 and 3 pages. You can also get some inspiration from papers published at top-tier conferences, although they are of course \emph{much} shorter due to space constraints.

Some people prefer ending the introduction with a paragraph that gives an overview of the following chapters. However, this is more usual for scientific papers (published at conferences or in journals) but not in project/thesis reports since they have a table of contents anyway. You are still free to add one if you prefer.
